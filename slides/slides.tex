\documentclass{beamer}

\usefonttheme{professionalfonts}
\usefonttheme{serif}
\usepackage{inputenc}[utf8]
\usepackage{fontspec}
\setmainfont{Libertinus Serif}
\usepackage{unicode-math}
\setmathfont{Libertinus Math}
\usepackage{xeCJK}
\setCJKmainfont{Noto Serif CJK TC}
\usepackage[backend=biber, sorting=ynt, style=verbose-ibid]{biblatex}
\addbibresource{slides.bib}
\usepackage{graphicx}
\graphicspath{{../pics}}
\usepackage{epigraph}
\usepackage{minted}
\setminted[sml]{autogobble=true, mathescape=true, escapeinside=::}
\usepackage{tcolorbox}
\usepackage{colortbl}
\usepackage{xcolor}
\usepackage{tipa}
\usepackage{amsmath}
\usepackage{amssymb}

\usetheme{CambridgeUS}

\title{Automated Deduction in Historical Phonology}
\author{陳朝陽 \\ Chen Zhaoyang \\ \texttt{zc23@illinois.edu}}

\begin{document}

\maketitle

\begin{frame}
	\frametitle{\textsc{tl; dr}}

	I wrote a small piece of program ($\sim$ 1,000 loc) that implements a couple dozen major phonological changes from Latin to Modern Spanish. The program is able to derive a good amount of Modern Spanish words from their Latin etymon. \\
	This presentation is about how it is done.
\end{frame}

\begin{frame}
	\frametitle{Reviving a field that has been silent for a while}

    Early attempts of works alike goes back the late '60s\footcites{williams_mechanise_phone}{smith69:_autom_simul_histor_chang}. Previous works on automated forward reconstruction has been done on Romance languages a couple times e.g. for Old French\footcite{burton-hunter76:_roman_etymol}, and Spanish\footcite{eastlack77:_progr_simul_system_sound_chang_ibero_roman}. \\
    As of now, this field has been silent since the late '90s.
\end{frame}

\begin{frame}
    \frametitle{Why do it again?}
  
	An essentially identical project was done from the 1980s to early 2000s by S. Lee Hartman and is still accessible online as of now \footcite{hartman_phono}. \\
	So why do it again? \\
	The answer is that, sadly, this is field largely inactive as of now; many works were done in a time when publishing programs were not feasible, thus many works were simply lost. \\
	Luckily, computing has made enough progress so that this work could be done rather quickly in a high-level programming language.
\end{frame}

\begin{frame}
	\frametitle{The Standard ML Programming Language}

	This small system is written entirely in Standard ML. \\
	SML is a relatively small and concise programming language that has been primarily used in programming language implementation and automated theorem proving. \\
	Its \emph{Heimatland} was University of Edinburgh, frist implemented in the LCF (logic of computable functions) theorem prover\footcite{macqueen20:_histor_stand_ml} by Robin Milner and his collegues, whose predecessor is the Stanford LCF system, written in LISP, also by R. Milner. \\
\end{frame}

\begin{frame}
	\frametitle{Implementing a Phonology}

	This project has a simple two-layer structure: the first layer that defines ways of constructing the \emph{statics} of a phonology -- namely the segmental inventory, syllable structure, and the phonological word of the Spanish language and her predecessors -- and the second layer that defines ways of constructing the \emph{dynamics} of a phonology -- namely sound changes and how to compose them into chain shifts.
\end{frame}

\begin{frame}
	\frametitle{The Statics}
	\framesubtitle{Representing Features, Segments, Syllables, and the Phonological Word}

    We define all of the \emph{static} element of this system through recursive types in ML. \\
    Recursive types are basically a combination of \textbf{and} \& \textbf{or}'s.
\end{frame}

\begin{frame}
	\frametitle{The Dynamics}
	\framesubtitle{Rewriting of Syllabic Constituents, Syllables, and Phonological Words}

    ML provides extensive supports for composing functions.

\end{frame}

\begin{frame}[fragile]
    \begin{minted}{sml}
      
      (* rewrites that changes
      a constituent of the syllable *)

      datatype rewrite = Onsetism of (onset -> onset)
                       | Nucleusism of (nucleus -> nucleus)
                       | Codism of (coda -> coda)

      (* syllabism that rewrites a syllable to another *)

      datatype context = NoContext
                       | Predicate of (syllable -> bool)
                       | WordInit
                       | WordFinal

      datatype syllabism
      = Syllabism of (syllable -> syllable) * context

    \end{minted}
\end{frame}

\begin{frame}
	\frametitle{Pearls of Sound Changes from Latin to Romance}

	In the remainder of this presentation, I am going to demonstrate 6 sets of sound changes to demonstrate how this system works. The majority of the words in examples comes from \emph{Romance Languages: A Historical Introduction}\footcite{romance_his}, \citetitle{lloyd_spanish}\footcite{lloyd_spanish}, and \citetitle{penny_spanish}\footcite{penny_spanish}.

\end{frame}

\begin{frame}
	\frametitle{Latin $\rightarrow$ Proto-Romance: \\ Romance Vowel Shifts}

	Arguably the most fundamental change from Late Latin to Proto-Romance is the transformation of its vowel system. \\
	The transformation has the following components:
	\begin{enumerate}
		\item Loss of Vowel Quantity
		\item The Great Vowel Merger
		\item Merger in Atonic Vowels
	\end{enumerate}
	Another important sound change, the loss of hiatus, unfortunately we are not going to cover in this presentation.
\end{frame}

\begin{frame}
	\frametitle{Loss of Vowel Quantity}

	\begin{tabular}{|c|c|c|c|c|c|c|}
		\hline
		     & \multicolumn{2}{c|}{Front}      & \multicolumn{2}{c|}{Cent.} & \multicolumn{2}{c|}{Back}                                                                            \\
		\hline
		High & \cellcolor{gray} \textsc{\u{i}} & \textsc{\={i}}             &                                 &                & \cellcolor{gray}  \textsc{\u{u}} & \textsc{\={u}} \\
		\hline
		Mid  & \cellcolor{gray} \textsc{\u{e}} & \textsc{\={e}}             &                                 &                & \cellcolor{gray}  \textsc{\u{o}} & \textsc{\={o}} \\
		\hline
		Low  &                                 &                            & \cellcolor{gray} \textsc{\u{a}} & \textsc{\={a}} &                                  &                \\
		\hline
	\end{tabular}
	\vspace{0.2cm}
	\begin{tabular}{|c|c|c|c|}
		\hline
		         & Front                           & Central & Back                            \\
		\hline
		High     & i                               &         & u                               \\
		\hline
		High-Mid & \cellcolor{magenta} \textipa{I} &         & \cellcolor{magenta} \textipa{U} \\
		\hline
		Mid      & e                               &         & o                               \\
		\hline
		Low-Mid  & \cellcolor{magenta} \textipa{E} &         & \cellcolor{magenta} \textipa{O} \\
		\hline
		Low      &                                 & a       &                                 \\
		\hline
	\end{tabular}

\end{frame}

\begin{frame}
	\begin{center}
		\begin{tabular}{c c}
			\textsc{latina}                      & Español                     \\
			\hline
			\textsc{v\textcolor{red}{\={i}}ta}   & v\textcolor{magenta}{i}da   \\
			\textsc{vic\textcolor{red}{\={i}}na} & vec\textcolor{magenta}{i}na \\
			\textsc{far\textcolor{red}{\={i}}na} & har\textcolor{magenta}{i}na \\
			\textsc{l\textcolor{red}{\={u}}na}   & l\textcolor{magenta}{u}na   \\
			\textsc{d\textcolor{red}{\={u}}ra}   & d\textcolor{magenta}{u}ra   \\
			\textsc{m\textcolor{red}{\={u}}ru}   & m\textcolor{magenta}{u}ro   \\
			\textsc{h\textcolor{red}{\={o}}ra}   & h\textcolor{magenta}{o}ra   \\
			\textsc{c\textcolor{red}{\={o}}rte}  & c\textcolor{magenta}{o}rte  \\
			\textsc{d\textcolor{red}{\={e}}bet}  & d\textcolor{magenta}{e}be   \\
			\textsc{t\textcolor{red}{\={e}}rnu}  & t\textcolor{magenta}{e}rno  \\
		\end{tabular}
	\end{center}

\end{frame}


\begin{frame}
	\frametitle{The Great Merger}

	\begin{tabular}{|c|c|c|c|}
		\hline
		         & Front                        & Central & Back                         \\
		\hline
		High     & i                            &         & u                            \\
		\hline
		High-Mid & \cellcolor{gray} \textipa{I} &         & \cellcolor{gray} \textipa{U} \\
		\hline
		Mid      & e                            &         & o                            \\
		\hline
		Low-Mid  & \textipa{E}                  &         & \textipa{O}                  \\
		\hline
		Low      &                              & a       &                              \\
		\hline
	\end{tabular}
	\vspace{0.2cm}
	\begin{tabular}{|c|c|c|c|}
		\hline
		        & Front                 & Central & Back                  \\
		\hline
		High    & i                     &         & u                     \\
		\hline
		Mid     & \cellcolor{magenta} e &         & \cellcolor{magenta} o \\
		\hline
		Low-Mid & \textipa{E}           &         & \textipa{O}           \\
		\hline
		Low     &                       & a       &                       \\
		\hline
	\end{tabular}
\end{frame}

\begin{frame}
	\begin{tabular}{c c c}
		\textsc{latina}                                  & PrRom         & Español                                        \\
		\hline
		\textsc{g\textcolor{red}{u}la}                   & [\textipa{U}] & g\textcolor{magenta}{o}la                      \\
		\textsc{c\textcolor{red}{u}rrit}                 & [\textipa{U}] & c\textcolor{magenta}{o}rre                     \\
		\textsc{m\textcolor{red}{\u{u}}sca}              & [\textipa{U}] & m\textcolor{magenta}{o}sca                     \\
		\textsc{b\textcolor{red}{i}b\textcolor{red}{i}t} & [\textipa{I}] & b\textcolor{magenta}{e}b\textcolor{magenta}{e} \\
		\textsc{l\textcolor{red}{i}ttera}                & [\textipa{I}] & l\textcolor{magenta}{e}tra                     \\
		\textsc{v\textcolor{red}{\u{i}}ce}               & [\textipa{I}] & v\textcolor{magenta}{e}z                       \\
	\end{tabular}
\end{frame}

\begin{frame}
	\frametitle{Atonic Merger}

	ɛ $\rightarrow$ e \\
	ɔ $\rightarrow$ o \\
	We can see that the Latin five vowel system is basically restored after these changes (after stressed [ɛ] and [ɔ] diphthongizes in Castilian, it is indeed fully restored.)
\end{frame}

\begin{frame}
	\begin{center}
		\begin{tabular}{c c}
			\textsc{lat.}                            & Es.                      \\
			\hline
			\textsc{h\={i}bern\textcolor{red}{u}}    & iviern\textcolor{red}{o} \\
			\textsc{c\textcolor{red}{i}rc\={a}re}    & c\textcolor{red}{e}rcar  \\
			\textsc{v\={e}n\={a}t\textcolor{red}{u}} & vena\textcolor{red}d{o}  \\
		\end{tabular}
	\end{center}
\end{frame}

\begin{frame}
	\frametitle{Romance Vowel Shifts: Rule Ordering}

	Loss of Quantity $<$ Great Merger $<$ Atonic Merger
\end{frame}

\begin{frame}
	\frametitle{Latin $\rightarrow$ Proto-Romance: \\ Fundamental Consonantal Shifts}

	m $\rightarrow \varnothing$ \\
	j $\sim$ ʝ $\rightarrow$ dʒ $\sim$ ɟ \\
	w $\rightarrow$ β \\
	tʲ $\rightarrow$ ts \\
	kʲ $\rightarrow$ tʃ \\
	dʲ, ɡʲ $\rightarrow$ dʒ $\sim$ ɟ
\end{frame}

\begin{frame}
	\frametitle{Proto-Romance $\rightarrow$ Western Romance: \\ Intervocalic Lenition I}

	\begin{center}
		\begin{tabular}{c c}
			\textsc{lat}                      & Es.                                        \\
			\hline
			\textsc{ca\textcolor{red}{b}allu} & ca\textcolor{magenta}{b}allo [\textipa{B}] \\
			\textsc{de\textcolor{red}{b}ere}  & de\textcolor{magenta}{b}er [\textipa{B}]   \\
			\textsc{ha\textcolor{red}{b}ere}  & ha\textcolor{magenta}{b}er [\textipa{B}]   \\
			                                  &                                            \\
			\textsc{cru\textcolor{red}{d}u}   & cru\textcolor{magenta}{d}o [\textipa{D}]   \\
			\textsc{pe\textcolor{red}{d}e}    & pie [$\varnothing$]                        \\
			                                  &                                            \\
			\textsc{au\textcolor{red}{g}ustu} & a\textcolor{magenta}{g}osto [\textipa{G}]  \\
			\textsc{li\textcolor{red}{g}are}  & li\textcolor{magenta}{g}ar [\textipa{G}]   \\
			\textsc{pa\textcolor{red}{g}anu}  & pa\textcolor{magenta}{g}ano [\textipa{G}]  \\
		\end{tabular}
	\end{center}

\end{frame}

\begin{frame}

	\begin{center}
		\begin{tabular}{c c}
			\textsc{lat.}                     & Es.                                         \\
			\hline
			\textsc{sa\textcolor{red}{p}ore}  & sa\textcolor{magenta}{b}or [\textipa{B}]    \\
			\textsc{ca\textcolor{red}{p}ut}   & ca\textcolor{magenta}{b}o [\textipa{B}]     \\
			\textsc{co\textcolor{red}{p}ertu} & cu\textcolor{magenta}{b}ierto [\textipa{B}] \\
			                                  &                                             \\
			\textsc{vi\textcolor{red}{t}a}    & vi\textcolor{magenta}{d}a [\textipa{D}]     \\
			\textsc{fa\textcolor{red}{t}a}    & ha\textcolor{magenta}{d}a [\textipa{D}]     \\
			\textsc{ca\textcolor{red}{t}ena}  & ca\textcolor{magenta}{d}ena [\textipa{D}]   \\
			                                  &                                             \\
			\textsc{ami\textcolor{red}{c}a}   & ami\textcolor{magenta}{g}a [\textipa{G}]    \\
			\textsc{se\textcolor{red}{c}uru}  & se\textcolor{magenta}{g}uro [\textipa{G}]   \\
			\textsc{fo\textcolor{red}{c}u}    & fue\textcolor{magenta}{g}o [\textipa{G}]    \\
		\end{tabular}
	\end{center}

\end{frame}

\begin{frame}

	\begin{center}
		\begin{tabular}{c c}
			\textsc{lat.}                      & Es.                            \\
			\hline
			\textsc{ser\textcolor{red}{p}ente} & ser\textcolor{magenta}{p}iente \\
			\textsc{al\textcolor{red}{p}es}    & a\textcolor{magenta}{p}les     \\
			\textsc{rum\textcolor{red}{p}ere}  & rum\textcolor{magenta}{p}er    \\
			                                   &                                \\
			\textsc{or\textcolor{red}{t}ica}   & or\textcolor{magenta}{t}iga    \\
			\textsc{men\textcolor{red}{t}a}    & men\textcolor{magenta}{t}a     \\
			                                   &                                \\
			\textsc{ar\textcolor{red}{c}u}     & ar\textcolor{magenta}{c}o      \\
			\textsc{fal\textcolor{red}{c}one}  & hal\textcolor{magenta}{c}ón    \\
		\end{tabular}
	\end{center}

\end{frame}

\begin{frame}
	\frametitle{Proto-Romance $\rightarrow$ Western Romance: \\ Degemination}

	\begin{center}
		\begin{tabular}{c c}
			\textsc{lat.}                                       & Es.                                             \\
			\hline
			\textsc{o\textcolor{red}{ss}u}                      & hue\textcolor{magenta}{s}o                      \\
			\textsc{su\textcolor{red}{mm}a}                     & su\textcolor{magenta}{m}a                       \\
			\textsc{a\textcolor{red}{pp}e\textcolor{red}{ll}at} & a\textcolor{magenta}{p}e\textcolor{magenta}{l}a \\
			\textsc{li\textcolor{red}{tt}era}                   & le\textcolor{magenta}{t}ra                      \\
			\textsc{si\textcolor{red}{cc}u}                     & se\textcolor{magenta}{c}o                       \\
		\end{tabular}
	\end{center}

\end{frame}

\begin{frame}
	\frametitle{Westerm Romance $\rightarrow$ Old Spanish: \\ Debuccalization of [ɸ]}

	\begin{center}
		\begin{tabular}{c c}
			\textsc{lat.}                     & Es.                         \\
			\hline
			\textsc{\textcolor{red}{f}ilu}    & \textcolor{magenta}{h}ilo   \\
			\textsc{\textcolor{red}{f}erire}  & \textcolor{magenta}{h}erir  \\
			\textsc{\textcolor{red}{f}erro}   & \textcolor{magenta}{h}ierro \\
			\textsc{\textcolor{red}{f}alcone} & \textcolor{magenta}{h}alcón \\
			                                  &                             \\
			\textsc{\textcolor{red}{f}ocu}    & \textcolor{magenta}{f}uego  \\
			\textsc{\textcolor{red}{f}ora}    & \textcolor{magenta}{f}uera  \\
			\textsc{\textcolor{red}{f}onte}   & \textcolor{magenta}{f}uente \\
			\textsc{\textcolor{red}{f}ronte}  & \textcolor{magenta}{f}rente \\
			\textsc{\textcolor{red}{f}lore}   & \textcolor{magenta}{f}lor   \\
			\textsc{\textcolor{red}{f}laccu}  & \textcolor{magenta}{f}laco  \\
		\end{tabular}
	\end{center}

\end{frame}

\begin{frame}
	\frametitle{Old Spanish $\rightarrow$ Modern Spanish: \\ The Spanish Sibilant Rearrangement}

	The sibilants in Alfonsino Spanish:
	\begin{center}
		\begin{tabular}{c c c c}
			          & dental affricate                                 & alveolo-apical & palatal                                                \\
			\hline
			voiceless & \c{c} [\textipa{\texttslig}] ($\rightarrow$ [s̪]) & -ss- [s̺]       & x [ʃ]                                                  \\
			voiced    & z [\textipa{\textdzlig}] ($\rightarrow$ [z̪])     & -s- [z̺]        & j, ge, gi [ʒ] ($\leftarrow$ [\textipa{\textdyoghlig}]) \\
		\end{tabular}
	\end{center}
\end{frame}

\begin{frame}
	\frametitle{Bibliography}

	\printbibliography
\end{frame}

\end{document}
