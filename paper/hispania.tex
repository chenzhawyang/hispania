\documentclass{report}

\usepackage{geometry}[a4paper]
\usepackage{appendix}[toc]
\usepackage{xeCJK}
\setCJKmainfont{Source Han Serif}
\usepackage{inputenc}[utf-8]
\usepackage{fontspec}
\setmainfont{FreeSerif}
\usepackage{babel}[USenglish]
\usepackage{minted}
\usepackage{bussproofs}
\usepackage{qtree}
\usepackage{amsmath}
\usepackage{amssymb}
\usepackage{amsthm}
\usepackage{tikz-cd}
\usepackage{tipa}
\usepackage{listings}
\usepackage[backend=biber, sorting=none, style=alphabetic]{biblatex}
\addbibresource{hispania.bib}
\usepackage{xcolor}
\usepackage{yfonts}
\usepackage{footnote}
\makesavenoteenv{tabular}

\setlength\parindent{0pt}

\newtheorem{definition}{Definition}
\newtheorem{theorem}{Theorem}
\newtheorem{lemma}{Lemma}
\newtheorem{remark}{Remark}

\title{Semi-Automated Language Changes \\ from Latin to Spanish}
\author{Chen Zhaoyang \\ \texttt{zc23@illinois.edu}}

\begin{document}

\maketitle

\pagebreak

\begin{abstract}

  This paper talks about an implementation of rule-based phonology in a high-level programming language.

\end{abstract}

\pagebreak

\tableofcontents

\pagebreak

\chapter{Glossary}

\section{Nominal}

\subsection{Declension \textsc{i}}

\begin{tabular}{|c|c|}
  \hline
  \textsc{latina} & Español \\
  \hline
  \textsc{capra} & cabra \\
  \hline
\end{tabular}

\subsection{Declension \textsc{ii}}

\begin{tabular}{|c|c|}
  \hline
  \textsc{latina} & Español \\
  \hline
  \textsc{murus} & muro \\
  \hline
\end{tabular}

\subsection{Declension \textsc{iii}}

\begin{tabular}{|c|c|}
  \hline
  \textsc{latina} & Español \\
  \hline
  \textsc{panis} & pan \\
  \hline
\end{tabular}

\subsection{Declension \textsc{iv}}

\begin{tabular}{|c|c|}
  \hline
  \textsc{latina} & Español \\
  \hline
  \textsc{fructus} & fructo\footnote{Old Spanish \emph{frucho}} \\
  \hline
\end{tabular}

\subsection{Declension \textsc{v}}

\begin{tabular}{|c|c|}
  \hline
  \textsc{latina} & Español \\
  \hline
  \textsc{dies} & día \\
  \hline
\end{tabular}

\section{Pronoun}

\begin{tabular}{|c|c|}
  \hline
  \textsc{latina} & Español \\
  \hline
  \textsc{ego} & yo \\
  \hline
  \textsc{tv} & tú \\
  \hline
  \textsc{nos} & nosotros\footnote{Old Spanish \emph{nos}} \\
  \hline
  \textsc{vos} & vosotros\footnote{Old Spanish \emph{vos}} \\
  \hline
  \textsc{ille} & él \\
  \hline
  \textsc{illi} & ellos \\
  \hline
\end{tabular}

\section{Numeral}

\begin{tabular}{|c|c|}
  \hline
  \textsc{latina} & Español \\
  \hline
  \textsc{vnvs} & uno \\
  \hline
  \textsc{dvo} & dos \\
  \hline
  \textsc{tres} & tres \\
  \hline
  \textsc{qvattvor} & cuatro \\
  \hline
  \textsc{qvinqve} & cinco \\
  \hline
  \textsc{sex} & seis \\
  \hline
  \textsc{septem} & siete \\
  \hline
  \textsc{octo} & ocho \\
  \hline
  \textsc{novem} & nueve \\
  \hline
  \textsc{decem} & diez \\
  \hline
\end{tabular}

\section{Verbal}

\subsection{Conjugation \textsc{i}}

\begin{tabular}{|c|c|}
  \hline
  \textsc{latina} & Español \\
  \hline
  \textsc{pensare} & pensar \\
  \hline
\end{tabular}

\subsection{Conjugation \textsc{ii}}

\begin{tabular}{|c|c|}
  \hline
  \textsc{latina} & Español \\
  \hline
  \textsc{habere} & haber \\
  \hline
\end{tabular}

\subsection{Conjugation \textsc{iii}}

\begin{tabular}{|c|c|}
  \hline
  \textsc{latina} & Español \\
  \hline
  \textsc{perdere} & perder \\
  \hline
\end{tabular}

\subsection{Conjugation \textsc{iv}}

\begin{tabular}{|c|c|}
  \hline
  \textsc{latina} & Español \\
  \hline
  \textsc{partire} & partir \\
  \hline
\end{tabular}

\chapter{Representation of Segments}

\section{Vocalic}

\begin{lstlisting}[basicstyle=\ttfamily, mathescape, escapeinside={(:}{:)}]
  vowel ::= height (:$\times$:) centrality (:$\times$:) duration

  height ::= low | low-mid | mid | high-mid | high

  centrality ::= front | central | back

  duration ::= short | long
\end{lstlisting}

\begin{minted}[autogobble]{hs}
  data Vowel = Vowel
  { height :: Height
    , centrality :: Centrality
    , dure :: Duration
  }

  data Height
  = Low
  | LoMid
  | Mid
  | HiMid
  | High

  data Centrality
  = Front
  | Cent
  | Back

  data Duration
  = Short
  | Long
\end{minted}

\begin{tabular}{|c|c|c|c|}
  \hline
  & Front & Central & Back \\
  \hline
  High & \textsc{i i:} & & \textsc{u u:} \\
  \hline
  Mid & \textsc{e e:} & & \textsc{o o:} \\
  \hline
  Low & & \textsc{a a:} & \\
  \hline
\end{tabular}

\section{Consonantal}

\begin{lstlisting}[basicstyle=\ttfamily, mathescape, escapeinside={(:}{:)}]
  consonant ::= manner (:$\times$:) place (:$\times$:) voice

  manner ::= nasal | stop | fricative | affricate
           | approximant | tap | trill | lateral

  place ::= bilabial
          | dental
          | palatal
          | palatoalvelar | velar | labialvelar
          | glottal

  place ::= voiced | voiceless
\end{lstlisting}

\chapter{Representation of Syllables}

\section{Onset}

\subsection{Onset Cluster}

\section{Nucleus}

\section{Coda}

\subsection{Coda Cluster}

\section{Syllable Weight and Stress}

\chapter{Representation of Phonotactics and Sound Changes}

\section{Phonotactics}

\section{Cardinality in String Rewriting}

\section{Context and Admissibility}

\section{Semantics of Sound Changes}

\chapter{Representation of Morphophonology}

\section{Nominal}

\section{Pronoun}

\section{Verbal}

\chapter{Anthology of Sound Changes}

\section{Vocalism}

Latin \\

\begin{tabular}{|c|c|c|c|}
  \hline
  & Front & Central & Back \\
  \hline
  High & \textsc{i i:} & & \textsc{u u:} \\
  \hline
  Mid & \textsc{e e:} & & \textsc{o o:} \\
  \hline
  Low & & \textsc{a a:} & \\
  \hline
\end{tabular}

Pre-Romance \\

\begin{tabular}{|c|c|c|c|}
  \hline
  & Front & Central & Back \\
  \hline
  High & i & & u \\
  \hline
  High-Mid & \textipa{I} & & \textipa{U} \\
  \hline
  Mid & e & & o \\
  \hline
  Low-Mid & \textipa{E} & & \textipa{O} \\
  \hline
  Low & & a & \\
  \hline
\end{tabular}

Proto-Romance after the Great Merger \\

\begin{tabular}{|c|c|c|c|}
  \hline
  & Front & Central & Back \\
  \hline
  High & i & & u \\
  \hline
  Mid & e & & o \\
  \hline
  Low-Mid & \textipa{E} & & \textipa{O} \\
  \hline
  Low & & a & \\
  \hline
\end{tabular}

\subsection{Pre-Romance Loss of Quantity}

\begin{equation}
\textbf{V}
  \begin{cases}
    \vdash \text{long} : \textsc{duration} & \Rightarrow [\text{long} \rightarrow \text{short}]\ \textbf{V} \\
    \vdash \text{short} : \textsc{duration} & \Rightarrow
                   \begin{cases}
                     \textsc{a} & \Rightarrow \textbf{V} \\
                     \text{else} & \Rightarrow \begin{cases}
                                                 \vdash \text{high} : \textsc{height} & \Rightarrow [\text{high} \rightarrow \text{high-mid}]\ \textbf{V} \\
                                                 \vdash \text{mid} : \textsc{height} & \Rightarrow [\text{mid} \rightarrow \text{low-mid}]\ \textbf{V} \\
                                               \end{cases} \\
                   \end{cases} \\
  \end{cases}
\end{equation}

\subsection{Pan-Romance Great Merger}

\begin{equation}
  \textbf{V}
  \begin{cases}
    \vdash \text{high-mid} : \textsc{height} & \Rightarrow [\text{high-mid} \rightarrow \text{mid}]\ \textbf{V} \\
    \vdash \text{else} & \Rightarrow \textbf{V}
  \end{cases}
\end{equation}

\subsection{Monophthongization}

\begin{align}
\textsc{oe} & \Rightarrow \text{e} \\
\textsc{au} & \Rightarrow \text{o} \\
\textsc{ae} & \Rightarrow \text{\textipa{E}}\footnote{sometimes [e]}
\end{align}
  
\subsection{Diphthongization}

\begin{equation}
  \textbf{V} \vdash \text{low-mid} : \textsc{height}
  \begin{cases}
    \vdash \text{front} : \textsc{centrality} & \Rightarrow \text{ie} \\
    \vdash \text{back} : \textsc{centrality} & \Rightarrow \text{we} \\
  \end{cases}
\end{equation}

\subsection{Loss of Hiatus}

\begin{equation}
  \textbf{V.V} \Rightarrow \textbf{VV}
\end{equation}

\subsection{Metaphony}

\section{Consonantism}

\subsection{Prothesis in /sC/}

\begin{equation}
  \#\ \text{s}\textbf{C} \Rightarrow \text{es}.\textbf{C}
\end{equation}

\subsection{Syncope}

\begin{equation}
  \textbf{CV.C} \Rightarrow \textbf{CC}
\end{equation}

\subsection{Betacism}

\begin{align}
\text{w} & \Rightarrow \text{\textipa{B}} \\
\text{b} & \Rightarrow \text{\textipa{B}}
\end{align}

\subsection{Diphthongs from Betacism}

\begin{equation}
\textbf{V}\text{\textipa{B}} \Rightarrow \textbf{V}\text{w}
\end{equation}

\subsection{Deaspiration in Popular Latin}

\begin{equation}
  \text{h} \Rightarrow \varnothing
\end{equation}

\subsection{Cluster Reduction in /ns/}

\begin{equation}
  \text{ns} \Rightarrow \text{s}
\end{equation}

\subsection{Elision of intervocalic /g/}

\begin{equation}
  \textbf{V}.\text{g}\textbf{V} \Rightarrow \textbf{VV}
\end{equation}

\subsection{Elision of coda /m/}

\begin{equation}
  \text{m}\ \# \Rightarrow \varnothing
\end{equation}

\subsection{Degemination}

\begin{equation}
  \textbf{CC} \Rightarrow \textbf{C}
\end{equation}

\subsection{Intervocalic Lenition}

\begin{align}
  \textbf{V.C}_{[-\text{voice}]}\textbf{V} & \Rightarrow \textbf{V.C}_{[+\text{voice}]}\textbf{V} \\
  \textbf{V.C}\substack{[+\text{voice}] \\ [+\text{stop}]}\textbf{V} & \Rightarrow \textbf{V.C}\substack{[+\text{voice}] \\ [+\text{fricative}]}\textbf{V}
\end{align}

\subsection{Debuccalization of onset /f/}

\begin{equation}
  \#\ \text{f} \Rightarrow \text{h}
\end{equation}

\subsection{Fortition of onset /j/}

\begin{equation}
  \#\ \text{j} \Rightarrow \text{\textipa{\textdyoghlig}}
\end{equation}

\subsection{Palatalization of Consonant Clusters}

\begin{align}
  \text{kt} & \Rightarrow \text{it} \\
  \text{kl} & \Rightarrow \text{\textipa{L}}\footnotemark \\
  \text{n:} & \Rightarrow \text{\textipa{\textltailn}} \\
  \text{l:} & \Rightarrow \text{\textipa{L}} \\
  \text{di} & \Rightarrow \text{\textipa{\textdyoghlig}} \\
  \text{gi} & \Rightarrow \text{\textipa{\textdyoghlig}}
\end{align}
\footnotetext{is it so?}

\subsection{Palatalization of Velars}

\begin{equation}
  \textbf{C}_{[+\text{velar}]}\textbf{V}_{[+\text{front}]} \Rightarrow \textbf{C}_{[+\text{palatal}]}\textbf{V}_{[+\text{front}]}
\end{equation}


\subsection{Palatalization of Dentals}

\begin{equation}
  \textbf{C}\substack{[+\text{dental}] \\ [+\text{stop}]}\text{i} \Rightarrow \textbf{C}\substack{[+\text{dental}] \\ [+\text{affricate}]}
\end{equation}

\subsection{Palatalization of /n/ and /j/}

\begin{align}
  \text{nj} & \Rightarrow \text{\textipa{\textltailn}} \\
  \text{lj} & \Rightarrow \text{\textipa{L}}
\end{align}

\subsection{Palatalization of /Cl/ Clusters}

\begin{align}
  \#\ \textbf{C}\text{l} & \Rightarrow \text{\textipa{L}} \\
  \text{-}\textbf{C}\text{l-} & \Rightarrow \begin{cases}
                                              \text{\textipa{\textteshlig}} \\
                                              \text{\textipa{\textdyoghlig}} \\
                                              \text{\textipa{\textltailn}} \\
                                              \text{\textipa{L}}
                                            \end{cases}
\end{align}

\subsection{Evolution of /skl/, /sl/, and /tl/}

\begin{align}
  \text{skl} & \Rightarrow \begin{cases}
                             \text{\textipa{\texttslig}kl} \\
                             \text{\textipa{\textteshlig}}
                           \end{cases}\\
  \text{sl} & \Rightarrow \text{skl} \\
  \text{tl} & \Rightarrow \text{kl}
\end{align}

\chapter{Anthology of Morphophonological Changes}

\nocite{*}

\printbibliography

\end{document}
