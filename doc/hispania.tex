\documentclass{report}

\usepackage{geometry}[a4paper]
\usepackage{appendix}[toc]
\usepackage{xeCJK}
\setCJKmainfont{Source Han Serif}
\usepackage{inputenc}[utf-8]
\usepackage{fontspec}
\setmainfont{Noto Serif}
\usepackage{minted}
\usepackage{bussproofs}
\usepackage{qtree}
\usepackage{amsmath}
\usepackage{amssymb}
\usepackage{amsthm}
\usepackage[USenglish]{babel}

\newtheorem{definition}{Definition}
\newtheorem{theorem}{Theorem}
\newtheorem{lemma}{Lemma}
\newtheorem{remark}{Remark}

\usepackage{tikz-cd}
\usepackage{tipa}
\usepackage{listings}
\usepackage[backend=biber, sorting=none, style=alphabetic, maxcitenames = 2]{biblatex}
\addbibresource{hispania.bib}
\DefineBibliographyStrings{USenglish}{
  andothers = {\textbf{et al}\adddot}
}
\usepackage{xcolor}
\usepackage{epigraph}
\usepackage{tcolorbox}
\usepackage{colortbl}
\usepackage{mathtools}
\usepackage{stmaryrd}
\usepackage{fourier}
\usepackage{footnote}
\makesavenoteenv{tabular}
\makesavenoteenv{table}
\usepackage{titlesec}
% \renewcommand{\thechapter}{\Roman{chapter}}
% \titleformat{\chapter}[display]{\huge\bf}{\textsc{Capitulum} \thechapter}{1em}{}
\usepackage{tocloft}
\setlength\cftchapnumwidth{3em}
\setlength\cftsecnumwidth{3em}
\setlength\cftsubsecnumwidth{4em}
\usepackage{hyperref}
\hypersetup{
  colorlinks=true,
  linkcolor=cyan,
  filecolor=magenta,
}

\setlength\parindent{0pt}
% \pagenumbering{Roman}

\title{Semi-Automated Language Changes from Latin to Spanish \\ or, \emph{El sueño de mi juventud: Fonología automatizada}}
\author{陳朝陽 \\ Chen Zhaoyang \\ \texttt{zc23@illinois.edu}}

\begin{document}

\maketitle

% \pagebreak

% \hspace{0pt}
% \vfill

% \begin{center}
%   Dedicated to Victoria
% \end{center}

% \hspace{0pt}
% \vfill

% \pagebreak

% \hspace{0pt}
% \vfill

% \begin{center}
%   \begin{flushleft}
%   Cansado estoy de llamar; \\
%   mi garganta se ha enronquecido; \\
%   Han desfallecido mis ojos esperando a mi Dios. \\
%   \end{flushleft}
%   \vspace{0.5cm}
%   \begin{flushright}
%     Salmos 69:3
%   \end{flushright}
% \end{center}

% \begin{center}
%   \begin{flushleft}
%     危冠切浮雲\ 長劍出天外 \\
%     細故何足慮\ 高度跨一世 \\
%     非子為我馭\ 逍遙遊荒裔 \\
%     顧謝西王母\ 吾將從此逝 \\
%     豈與蓬戶士\ 彈琴誦言誓 \\
%   \end{flushleft}
%   \vspace{0.5cm}
%   \begin{flushright}
%     阮籍\ 詠懷第五十八\footnotemark
%   \end{flushright}
% \end{center}
% \footnotetext{\href{https://www.degruyter.com/document/doi/10.1515/9781501503870-004/pdf}{My towering cap touches the drifting clouds, my long sword sticks forth beyond the heavens.  Minor issues are not worth concern, passing on high, I stride across the whole age. Feizi is my carriage driver, free and easy, I roam at the wilderness’s edge. Looking back, I take leave of the Queen Mother, from this point on I will go off. How can I join the gentlemen in a thatched house, plucking a zither and reciting mutual pledges?} (Singing My Cares LVIII)}

% \vfill  
% \hspace{0pt}

% \pagebreak

\begin{abstract}
  
  This paper talks about an implementation of rule-based phonology in a high-level programming language. To illustrate the competence of the system, the system is used to implement major sound changes in the historical evolution of Latin to Modern Spanish. To a lesser extent, morphophonological changes from Latin to Spanish are also experimented. \\
  In the first couple chapters of this paper, we establish notations that represent the phonology and morphology of Latin and its Romance descendants following the tradition and convention of modern rule-based phonology that started to develop in the 1960s. After that we describe the major phonological and morphological changes the predecessors of Spanish had undergone, in the following historical periods: Proto-Romance, Western Romance, Central Iberian Romance, Old Spanish, Early Modern Spanish, and eventually we would be able to derive Modern Spanish. \\
  After settling the chronology and defining historical changes of Spanish language, in the last part of this project we explore the details of the implementation of a rewriting system that is able to substantiate and automate much of the aforementioned language changes.

\end{abstract}

\tableofcontents

\pagebreak

\chapter*{Introduction}
\addcontentsline{toc}{chapter}{Introduction}

% \epigraph{枯れた技術の水平思考\footnotemark}{横井軍平}
% \footnotetext{\href{https://en.wikipedia.org/wiki/Gunpei_Yokoi\#Lateral_Thinking_with_Withered_Technology}{``Lateral Thinking of Withered Technology.''}}

\section*{Withered Techniques}

\subsection*{Rule-based Phonology}

\subsection*{Romance Historical Linguistics}

\subsection*{Typed Programming Languages}

\chapter{Representation of Segments}

% \epigraph{En el principio era el Verbo, y el Verbo era con Dios, y el Verbo era Dios.}{Juan 1:1}

\section{Vocalic}

\begin{lstlisting}[basicstyle=\ttfamily, mathescape, escapeinside={(:}{:)}]
  vowel ::= height (:$\times$:) centrality (:$\times$:) duration

  height ::= low | low-mid | mid | high-mid | high

  centrality ::= front | central | back

  duration ::= short | long
\end{lstlisting}

\begin{minted}[autogobble]{hs}
  data Vowel = Vowel
  { height :: Height
    , centrality :: Centrality
    , dure :: Duration
  }

  data Height
  = Low
  | LoMid
  | Mid
  | HiMid
  | High

  data Centrality
  = Front
  | Cent
  | Back

  data Duration
  = Short
  | Long
\end{minted}

\section{Consonantal}

\begin{lstlisting}[basicstyle=\ttfamily, mathescape, escapeinside={(:}{:)}]
  consonant ::= manner (:$\times$:) place (:$\times$:) voice

  manner ::= nasal
           | stop
           | fricative
           | non-siblant fricative
           | affricate
           | approximant
           | tap
           | trill
           | lateral

  place ::= bilabial
          | labiodental
          | dental
          | alveolar
          | palatal
          | palatoalvelar | velar | labialvelar
          | glottal

  place ::= voiced | voiceless
\end{lstlisting}

\chapter{Representation of Syllables}

\section{Syllabic Structure}

\section{Syllable Weight and Stress}

\chapter{Representation of Morphophonology}

\section{Nominal}

\section{Verbal}

\chapter{Representation of Phonotactics and Sound Changes}

% \epigraph{星翳燈幻露泡夢電雲\footnotemark}{金剛經 \S32}
% \footnotetext{\href{https://www2.hf.uio.no/polyglotta/index.php?page=record\&vid=1176\&mid=1993243}{A shooting star, a clouding of the sight, a lamp, an illusion, a drop of dew, a bubble, a dream, a lightning’s flash, a thunder cloud.}}

\section{Sound Changes}

\section{Phonotactics}

\chapter{Proto-Romance}

% \epigraph{\textsc{finis origine pendent}}{\textsc{marcvs manilivs}}

\section{Phonemic Inventory}

\subsection{Vocalic}

\begin{tcolorbox}[hbox, title=Proto-Romance Monophthongs]
  \begin{tabular}{|c|c|c|c|}
    \hline
    & Front & Central & Back \\
    \hline
    High & i & & u \\
    \hline
    Mid & e & & o \\
    \hline
    Low-Mid & \textipa{E} & & \textipa{O} \\
    \hline
    Low & & a & \\
    \hline
  \end{tabular}
\end{tcolorbox}

\subsection{Consonantal}

\begin{tcolorbox}[title=Proto-Romance Consonants, hbox]
  \begin{tabular}{|c|c|c|c|c|c|c|c|c|}
    \hline
    & Bilabial & Labiodental & Dental & Alveolar & Palatal & Palatovelar & Velar & Labiovelar \\
    \hline
    Nasal & m & & \multicolumn{2}{c|}{n} & \textipa{\textltailn} & & & \\
    \hline
    Stop & p \quad b & & t \quad d & \textipa{t\super j} \quad \textipa{d\super j} & \textipa{\textbardotlessj} & \textipa{k\super j} \quad \textipa{g\super j} & k \quad g & \textipa{k\super w} \quad \textipa{g\super w} \\
    \hline
    Fricative & \textipa{B} & f & & & \textipa{J} & & & \\
    \hline
    \textquotedbl & & & & s \quad z & & & & \\
    \hline
    Approximant & & & & & j & & & \\
    \hline
    Trill & & & \multicolumn{2}{c|}{r} & & & & \\
    \hline
    Lateral & & & \multicolumn{2}{c|}{l} & \textipa{l\super j} & & & \\
    \hline
  \end{tabular}
\end{tcolorbox}

\section{Sound Changes}

\subsection{Vocalism}

\begin{tcolorbox}[title=Loss of Quantity, hbox]
  \begin{tabular}{c c}
    \begin{tabular}{|c|c|c|c|c|c|c|}
      \hline
      & \multicolumn{2}{c|}{Front} & \multicolumn{2}{c|}{Cent.} & \multicolumn{2}{c|}{Back} \\
      \hline
      High & \cellcolor{gray} \textsc{\u{i}} & \textsc{\={i}} & & & \cellcolor{gray} \textsc{\u{u}} & \textsc{\={u}} \\
      \hline
      Mid & \cellcolor{gray} \textsc{\u{e}} & \textsc{\={e}} & & & \cellcolor{gray} \textsc{\u{o}} & \textsc{\={o}} \\
      \hline
      Low &  &  & \cellcolor{gray} \textsc{\u{a}} & \textsc{\={a}} & & \\
      \hline
    \end{tabular}
    \quad $\Rightarrow$ & 
                          \begin{tabular}{|c|c|c|c|}
                            \hline
                            & Front & Central & Back \\
                            \hline
                            High & i & & u \\
                            \hline
                            High-Mid & \cellcolor{magenta} \textipa{I} & & \cellcolor{magenta} \textipa{U} \\
                            \hline
                            Mid & e & & o \\
                            \hline
                            Low-Mid & \cellcolor{magenta} \textipa{E} & & \cellcolor{magenta} \textipa{O} \\
                            \hline
                            Low & & a & \\
                            \hline
                          \end{tabular}
  \end{tabular}
\end{tcolorbox}

\begin{tabular}{c c}
  \textsc{latina} & Español \\
  \hline
  \textsc{v\textcolor{red}{\={i}}ta} & v\textcolor{magenta}{i}da \\
  \textsc{vic\textcolor{red}{\={i}}na} & vec\textcolor{magenta}{i}na \\
  \textsc{far\textcolor{red}{\={i}}na} & har\textcolor{magenta}{i}na \\
  \textsc{l\textcolor{red}{\={u}}na} & l\textcolor{magenta}{u}na \\
  \textsc{d\textcolor{red}{\={u}}ra} & d\textcolor{magenta}{u}ra \\
  \textsc{m\textcolor{red}{\={u}}ru} & m\textcolor{magenta}{u}ro \\
  \textsc{h\textcolor{red}{\={o}}ra} & h\textcolor{magenta}{o}ra \\
  \textsc{c\textcolor{red}{\={o}}rte} & c\textcolor{magenta}{o}rte \\
  \textsc{d\textcolor{red}{\={e}}bet} & d\textcolor{magenta}{e}be \\
  \textsc{t\textcolor{red}{\={e}}rnu} & t\textcolor{magenta}{e}rno \\
\end{tabular}

\begin{tcolorbox}[title=Great Merger, hbox]
  \begin{tabular}{c c}
    \begin{tabular}{|c|c|c|c|}
      \hline
      & Front & Central & Back \\
      \hline
      High & i & & u \\
      \hline
      High-Mid & \cellcolor{gray} \textipa{I} & & \cellcolor{gray} \textipa{U} \\
      \hline
      Mid & e & & o \\
      \hline
      Low-Mid & \textipa{E} & & \textipa{O} \\
      \hline
      Low & & a & \\
      \hline
    \end{tabular}
    \quad $\Rightarrow$ &
                          \begin{tabular}{|c|c|c|c|}
                            \hline
                            & Front & Central & Back \\
                            \hline
                            High & i & & u \\
                            \hline
                            Mid & \cellcolor{magenta} e & & \cellcolor{magenta} o \\
                            \hline
                            Low-Mid & \textipa{E} & & \textipa{O} \\
                            \hline
                            Low & & a & \\
                            \hline
                          \end{tabular}
  \end{tabular}
\end{tcolorbox}

\begin{tabular}{c c}
  \textsc{latina} & Español \\
  \hline
  \textsc{g\textcolor{red}{u}la} & g\textcolor{magenta}{o}la \\
  \textsc{c\textcolor{red}{u}rrit} & c\textcolor{magenta}{o}rre \\
  \textsc{m\textcolor{red}{\u{u}}sca} & m\textcolor{magenta}{o}sca \\
  \textsc{b\textcolor{red}{i}b\textcolor{red}{i}t} & b\textcolor{magenta}{e}b\textcolor{magenta}{e} \\
  \textsc{l\textcolor{red}{i}ttera} & l\textcolor{magenta}{e}tra \\
  \textsc{v\textcolor{red}{\u{i}}ce} & v\textcolor{magenta}{e}z \\
\end{tabular}

\begin{tcolorbox}[title=Monophthongization]
  \begin{align*}
    \textsc{oe} & \Rightarrow \text{e} \\
    \textsc{au} & \Rightarrow \text{o} \\
    \textsc{ae} & \Rightarrow \text{\textipa{E}}\ |\ \text{e} \\
  \end{align*}
\end{tcolorbox}

\begin{tabular}{c c}
  \textsc{latina} & Español \\
  \hline
  \textsc{p\textcolor{red}{oe}na} & p\textcolor{magenta}{e}na \\
  \textsc{f\textcolor{red}{oe}du} & f\textcolor{magenta}{e}o \\
  \textsc{\textcolor{red}{au}ru} & \textcolor{magenta}{o}ro \\
  \textsc{thes\textcolor{red}{au}ru} & tes\textcolor{magenta}{o}ro \\
  \textsc{p\textcolor{red}{au}peru} & p\textcolor{magenta}{o}bre \\
  \textsc{p\textcolor{red}{au}cu} & p\textcolor{magenta}{o}co \\
  \textsc{c\textcolor{red}{ae}cu} & c\textcolor{magenta}{ie}go \\
  \textsc{c\textcolor{red}{ae}lu} & c\textcolor{magenta}{ie}lo \\
  \textsc{s\textcolor{red}{ae}ta} & s\textcolor{magenta}{e}da \\
\end{tabular}

\begin{tcolorbox}[title=Syncope]
  
\end{tcolorbox}

\begin{tabular}{c c}
  \textsc{lat.} & Pop. Lat. \\
  \hline
  \textsc{oc\textcolor{red}{u}lu} & \textsc{oclu} \\
  \textsc{auric\textcolor{red}{u}la} & \textsc{oricla} \\
  \textsc{cal\textcolor{red}{i}du} & \textsc{caldu} \\
\end{tabular}

\begin{tabular}{c c}
  \textsc{lat.} & Es. \\
  \hline
  \textsc{lep\textcolor{red}{o}re} & liebre \\
  \textsc{i(n)s\textcolor{red}{u}la} & isla \\
  \textsc{vir\textcolor{red}{i}de} & verde \\
\end{tabular}

parab\textcolor{red}{o}la v. parabla \\

\begin{tcolorbox}[title=Apocope]

\end{tcolorbox}

\subsection{Consonantism}

\begin{tcolorbox}[hbox, title=Latin Consonants]
  \begin{tabular}{|c|c|c|c|c|c|c|c|c|}
    \hline
    & Bilabial & Labiodental & Dental & Alveolar & Palatal & Velar & Labiovelar & Glottal \\
    \hline
    Nasal & m & & \multicolumn{2}{c|}{n} & & & & \\
    \hline
    Stop & p \quad b & & t \quad d & & & k \quad g & \textipa{k\super w} \textipa{g\super w} & \\
    \hline
    Fricative & & f & & & & & & \cellcolor{gray} h \\
    \hline
    \textquotedbl & & & & s \quad z & & & & \\
    \hline
    Approximant & & & & & \cellcolor{gray} j & & \cellcolor{gray} w & \\
    \hline
    Trill & & & \multicolumn{2}{c|}{r} & & & & \\
    \hline
    Lateral & & & \multicolumn{2}{c|}{l} & & & & \\
    \hline
  \end{tabular}
\end{tcolorbox}

\begin{tcolorbox}[title=Prothesis in /sC/]

\end{tcolorbox}

\begin{tabular}{c c}
  \textsc{lat.} & Es. \\
  \hline
  \textsc{\textcolor{red}{sp}onsa} & \textcolor{magenta}{esp}osa \\
  \textsc{\textcolor{red}{sp}ata} & \textcolor{magenta}{esp}ada \\
  \textsc{\textcolor{red}{st}udiu} & \textcolor{magenta}{est}udio \\
  \textsc{\textcolor{red}{sp}ongia} & \textcolor{magenta}{esp}onja \\
  \textsc{\textcolor{red}{sc}utu} & \textcolor{magenta}{esc}udo \\
\end{tabular}

\begin{tcolorbox}[title=Nasal Liquid Cluster from Syncope]
  
\end{tcolorbox}

\begin{tabular}{c c}
  \textsc{lat.} & Es. \\
  \hline
  \textsc{te\textcolor{red}{ner}u} & tie\textcolor{magenta}{rn}o \\
  \textsc{ge\textcolor{red}{ner}u} & ye\textcolor{magenta}{rn}o \\
  \textsc{tre\textcolor{red}{mul}at} & tie\textcolor{magenta}{mbl}a \\
\end{tabular}

\begin{tcolorbox}[title=Nasal Nasal Cluster from Syncope]
  
\end{tcolorbox}

\begin{tcolorbox}[title=Betacism]

\end{tcolorbox}

\begin{tcolorbox}[title=Deaspiration]
  
\end{tcolorbox}

\begin{tcolorbox}[title=Nasal Spirant Law]

\end{tcolorbox}

\begin{tcolorbox}[title=Elision of Intervocalic /g/]

\end{tcolorbox}

\begin{tcolorbox}[title=Elision of Coda /m/]
  
\end{tcolorbox}

\section{Morphophonology}

\subsection{Nominal}

\subsubsection{Dissolution of Declension IV \& V}

\subsubsection{Reduction of the Case System}

\begin{tabular}{c c}
  \begin{tabular}{|c|c|c|}
    \hline
    \textsc{latina} & \textsc{sg} & \textsc{pl} \\
    \hline
    \textsc{nom} & \cellcolor{gray} \textsc{capra} & \cellcolor{yellow} \textsc{caprae} \\
    \hline
    \textsc{gen} & \textsc{caprae} & \textsc{capr\={a}rum} \\
    \hline
    \textsc{dat} & \textsc{caprae} & \textsc{capr\={i}s} \\
    \hline
    \textsc{acc} & \cellcolor{cyan} \textsc{capram} & \cellcolor{red} \textsc{capr\={a}s} \\
    \hline
    \textsc{abl} & \textsc{capr\={a}} & \textsc{capr\={i}s} \\
    \hline
    \textsc{voc} & \textsc{capra} & \textsc{caprae} \\
    \hline
  \end{tabular} \quad $\Rightarrow$ &
                                      \begin{tabular}{|c|c|c|}
                                        \hline
                                        Pop. Lat. & \textsc{sg} & \textsc{pl} \\
                                        \hline
                                        \textsc{nom} & \cellcolor{gray} \textsc{capra} & \cellcolor{yellow} \textsc{capre} \\
                                        \hline
                                        \textsc{obl} & \cellcolor{cyan} \textsc{capra} & \cellcolor{red} \textsc{capras} \\
                                        \hline
                                        \multicolumn{3}{c}{} \\
                                        \multicolumn{3}{c}{$\Downarrow$}
                                      \end{tabular} \\
  &
  \begin{tabular}{|c|c|c|}
    \hline
    Español & \textsc{sg} & \textsc{pl} \\
    \hline
    & \cellcolor{cyan} cabra & \cellcolor{red} cabras \\
    \hline
  \end{tabular} \\
\end{tabular}

\subsubsection{Reorganization of Genders}

\subsection{Verbal}

\chapter{Western Romance}

% \epigraph{\textsc{venite igitvr descendamvs et confvndamvs ibi lingvam eorvm vt non avdiat vnvsqvisqve vocem proximi svi}}{\textsc{genesis} \textsc{xi}$\bullet$\textsc{vii}}

\section{Phonemic Inventory}

\subsection{Vocalic}

\begin{tcolorbox}[title=Western Romance Monophthongs, hbox]
  \begin{tabular}{|c|c|c|c|}
    \hline
    & Front & Central & Back \\
    \hline
    High & i & & u \\
    \hline
    Mid & e & & o \\
    \hline
    Low & & a & \\
    \hline
  \end{tabular}
\end{tcolorbox}

\subsection{Consonantal}

\begin{tcolorbox}[title=Western Romance Consonants, hbox]
  \begin{tabular}{|c|c|c|c|c|c|c|c|}
    \hline
    & Bilabial & Labiodental & Dental & Alveolar & Palatal & Velar & Labiovelar \\
    \hline
    Nasal & m & & \multicolumn{2}{c|}{n} & \textipa{\textltailn} & & \\
    \hline
    Stop & p \quad b & & t \quad d & & & k \quad g & \textipa{k\super w} \quad \textipa{g\super w} \\
    \hline
    Fricative & \textipa{B} & f & \textipa{D} & & & \textipa{G} & \\
    \hline
    \textquotedbl & & & & s \quad z & & & \\
    \hline
    Affricate & & & \textipa{\texttslig} \quad \textipa{\textdzlig} & & \textipa{\textteshlig} \quad \textipa{\textdyoghlig} & & \\
    \hline
    Trill & & & \multicolumn{2}{c|}{r} & & & \\
    \hline
    Lateral & & & \multicolumn{2}{c|}{l} & \textipa{L} & & \\
    \hline
  \end{tabular}
\end{tcolorbox}

\section{Sound Changes}

\subsection{Vocalism}

\begin{tcolorbox}[title=Diphthongization]
  
\end{tcolorbox}

\begin{tabular}{c c c}
  \textsc{lat.} & Es. & Fr. \\
  \hline
  \textsc{p\textcolor{red}{e}tra} & p\textcolor{magenta}{ie}dra & p\textcolor{magenta}{ie}rre \\
  \textsc{f\textcolor{red}{e}le} & h\textcolor{magenta}{ie}l & f\textcolor{magenta}{ie}l \\
  \textsc{v\textcolor{red}{e}nit} & v\textcolor{magenta}{ie}ne & v\textcolor{magenta}{ie}nt \\
  \textsc{h\textcolor{red}{e}ri} & a\textcolor{magenta}{ye}r & h\textcolor{magenta}{ie}r \\
\end{tabular}

\subsection{Consonantalism}

\begin{tcolorbox}[title=Degemination]
  
\end{tcolorbox}

\begin{tcolorbox}[title=Palatalization of Dentals]

\end{tcolorbox}

\begin{tcolorbox}[title=Lenition I]

\end{tcolorbox}

\section{Morphophonology}

\subsection{Nominal}

\subsection{Verbal}

\chapter{Central Iberian Romance}

% \epigraph{Yo te aseguro que hoy estarás conmigo en el paraíso.}{Lucas 23:43}

\section{Phonemic Inventory}

\subsection{Vocalic}

\subsection{Consonantal}

\section{Sound Changes}

\subsection{Vocalism}

\begin{tcolorbox}[title=Diphthongization]
  
\end{tcolorbox}

\begin{tabular}{c c}
  \textsc{lat.} & Es. \\
  \textsc{hib\textcolor{red}{e}rnu} & inv\textcolor{magenta}{ie}rno \\
  \textsc{ap\textcolor{red}{e}rta} & ab\textcolor{magenta}{ie}rta \\
  \textsc{s\textcolor{red}{e}pte} & s\textcolor{magenta}{ie}te \\
  \textsc{c\textcolor{red}{e}rvu} & c\textcolor{magenta}{ie}rvo \\
  \textsc{f\textcolor{red}{e}rru} & h\textcolor{magenta}{ie}rro \\
  \textsc{f\textcolor{red}{o}rte} & f\textcolor{magenta}{ue}rte \\
  \textsc{p\textcolor{red}{o}rta} & p\textcolor{magenta}{ue}rta \\
  \textsc{m\textcolor{red}{o}rdit} & m\textcolor{magenta}{ue}rde \\
  \textsc{m\textcolor{red}{o}rit} & m\textcolor{magenta}{ue}re \\
  \textsc{m\textcolor{red}{o}vet} & m\textcolor{magenta}{ue}ve \\
  \textsc{p\textcolor{red}{o}tet} & p\textcolor{magenta}{ue}de \\
\end{tabular}

\begin{tcolorbox}[title=Raising]
  
\end{tcolorbox}

\begin{tabular}{c c}
  \textsc{lat.} & Es. \\
  \hline
  \textsc{m\textcolor{red}{u}ltu} & m\textcolor{magenta}{u}cho \\
  \textsc{ausc\textcolor{red}{u}ltat} & esc\textcolor{magenta}{u}cha \\
  \textsc{l\textcolor{red}{a}cte} & l\textcolor{magenta}{u}cha \\
  \textsc{l\textcolor{red}{a}cte} & l\textcolor{magenta}{e}che \\
  \textsc{f\textcolor{red}{a}ctu} & h\textcolor{magenta}{e}cho \\
  \textsc{b\textcolor{red}{a}siu} & b\textcolor{magenta}{e}so \\
  \textsc{c\textcolor{red}{a}seu} & q\textcolor{magenta}{ue}so \\
\end{tabular}

\subsection{Consonantalism}

\section{Morphophonology}

\subsection{Nominal}

\subsection{Verbal}

\chapter{Old Spanish}

% \epigraph{Ya lo vedes, que partirnos emos en vida, yo iré e vós fincaredes remanida.}{Cantar de Mio Cid}

\section{Phonemic Inventory}

\subsection{Vocalic}

\begin{tcolorbox}[title=Old Spanish Monophthongs, hbox]
  \begin{tabular}{|c|c|c|c|}
    \hline
    & Front & Central & Back \\
    \hline
    High & i & & u \\
    \hline
    Mid & e & & o \\
    \hline
    Low & & a & \\
    \hline
  \end{tabular}
\end{tcolorbox}

\subsection{Consonantal}

\begin{tcolorbox}[title=Old Spanish Consonants, hbox]
  \begin{tabular}{|c|c|c|c|c|c|c|c|}
    \hline
    & Bilabial & Dental & Alveolar & Palatal & Velar & Labiovelar & Glottal \\
    \hline
    Nasal & m & \multicolumn{2}{c|}{n} & \textipa{\textltailn} & & & \\
    \hline
    Stop & p \quad b & t \quad d & & & k \quad g & \textipa{k\super w} \quad \textipa{g\super w} & \\
    \hline
    Fricative & \textipa{F} \quad \textipa{B} & \textipa{D} & & \textipa{J} & \textipa{G} & & h \\
    \hline
    \textquotedbl & & & s \quad z & \textipa{S} \quad \textipa{Z} & & & \\
    \hline
    Affricate & & \textipa{\texttslig} \quad \textipa{\textdzlig} & & \textipa{\textteshlig} \quad \textipa{\textdyoghlig} & & & \\
    \hline
    Trill & & \multicolumn{2}{c|}{r} & & & & \\
    \hline
    Tap & & \multicolumn{2}{c|}{\textipa{R}} & & & & \\
    \hline
    Lateral & & \multicolumn{2}{c|}{l} & \textipa{L} & & & \\
    \hline
  \end{tabular}
\end{tcolorbox}

\section{Sound Changes}

\subsection{Vocalism}

\subsection{Consonantalism}

\section{Morphophonology}

\subsection{Nominal}

\subsection{Verbal}

\chapter{Early Modern Spanish}

% \epigraph{El que sirve una revolución ara en el mar.}{Simón Bolívar}

\section{Phonemic Inventory}

\subsection{Vocalic}

\subsection{Consonantal}

\section{Sound Changes}

\subsection{Vocalism}

\subsection{Consonantalism}

\section{Morphophonology}

\subsection{Nominal}

\subsection{Verbal}

\chapter{Modern Spanish}

% \epigraph{Todo está cumplido.}{Juan 19:30}

\section{Phonemic Inventory}

\subsection{Vocalic}

\subsection{Consonantal}

\section{Sound Changes}

\subsection{Vocalism}

\subsection{Consonantalism}

\section{Morphophonology}

\subsection{Nominal}

\subsection{Verbal}

% \chapter{Anthology of Sound Changes}

% % \epigraph{How many years must a mountain exist \\ Before it is washed to the sea?}{Bob Dylan}

% \section{Vocalism}

% \begin{tcolorbox}[hbox, title=Latin]
%   \begin{tabular}{|c|c|c|c|}
%     \hline
%     & Front & Central & Back \\
%     \hline
%     High & \textsc{\u{i} \={i}} & & \textsc{\u{u} \={u}} \\
%     \hline
%     Mid & \textsc{\u{e} \={e}} & & \textsc{\u{o} \={o}} \\
%     \hline
%     Low & & \textsc{\u{a} \={a}} & \\
%     \hline
%   \end{tabular}
% \end{tcolorbox} 

% \begin{tcolorbox}[hbox, title=Proto-Romance]
%   \begin{tabular}{|c|c|c|c|}
%     \hline
%     & Front & Central & Back \\
%     \hline
%     High & i & & u \\
%     \hline
%     High-Mid & \cellcolor{gray} \textipa{I} & & \cellcolor{gray} \textipa{U} \\
%     \hline
%     Mid & e & & o \\
%     \hline
%     Low-Mid & \textipa{E} & & \textipa{O} \\
%     \hline
%     Low & & a & \\
%     \hline
%   \end{tabular}
% \end{tcolorbox}

% \begin{tcolorbox}[hbox, title=Proto-Romance after the Great Merger]
%   \begin{tabular}{|c|c|c|c|}
%     \hline
%     & Front & Central & Back \\
%     \hline
%     High & i & & u \\
%     \hline
%     Mid & e & & o \\
%     \hline
%     Low-Mid & \cellcolor{gray} \textipa{E} & & \cellcolor{gray} \textipa{O} \\
%     \hline
%     Low & & a & \\
%     \hline
%   \end{tabular}
% \end{tcolorbox}

% \subsection{Monophthongs}

% \begin{tcolorbox}[title=Proto-Romance Loss of Quantity]  
%   \begin{equation}\tag{Loss of Quantity}\label{eq:loss_of_quantity}
%     \textbf{V}
%     \begin{cases}
%       \vdash \text{long} : \textsc{duration} & \Rightarrow [\text{long} \rightarrow \text{short}]\ \textbf{V} \\
%       \vdash \text{short} : \textsc{duration} & \Rightarrow
%                                                 \begin{cases}
%                                                   \textsc{a} & \Rightarrow \textbf{V} \\
%                                                   \texttt{else} & \Rightarrow \begin{cases}
%                                                                               \vdash \text{high} : \textsc{height} & \Rightarrow [\text{high} \rightarrow \text{high-mid}]\ \textbf{V} \\
%                                                                               \vdash \text{mid} : \textsc{height} & \Rightarrow [\text{mid} \rightarrow \text{low-mid}]\ \textbf{V} \\
%                                                                             \end{cases} \\
%                                                 \end{cases} \\
%     \end{cases}
%   \end{equation}  
% \end{tcolorbox}

% \begin{tcolorbox}[title=Pan-Romance Great Merger]
%   \begin{equation}\tag{Great Merger}\label{eq:great_merger}
%     \textbf{V}_{\vdash \text{high-mid}} \Rightarrow [\text{high-mid} \rightarrow \text{mid}]\ \textbf{V}
%   \end{equation}
% \end{tcolorbox}

% \begin{tcolorbox}[title=Monophthongization]
%   \begin{align*}\tag{Monophthongization}\label{eq:monophthongization}
%     \textsc{oe} & \Rightarrow \text{e} \\
%     \textsc{au} & \Rightarrow \text{o} \\
%     \textsc{ae} & \Rightarrow \text{\textipa{E}}\ |\ \text{e}
%   \end{align*}
% \end{tcolorbox}

% \begin{tcolorbox}[title=Diphthongization]
%   \begin{align*}\tag{Diphthongization}\label{eq:diphthongization}
%       \text{\textipa{E}} \in \mathbb{S}_{\vdash \text{stressed}} & \Rightarrow \text{ie} \\
%       \text{\textipa{O}} \in \mathbb{S}_{\vdash \text{stressed}} & \Rightarrow \text{we}
%   \end{align*}
% \end{tcolorbox}

% \subsection{Diphthongs}

% \begin{tcolorbox}[title=Loss of Hiatus]
%   \begin{equation}\tag{Loss of Hiatus}\label{eq:loss_of_hiatus}
%     \textbf{V.V} \Rightarrow \textbf{VV}
%   \end{equation}
% \end{tcolorbox}

% \subsection{Metaphony}

% \section{Consonantism}

% \subsection{Monolithic Consonants}

% \subsection{Consonant Clusters}

% \paragraph{Prothesis in /sC/}

% \begin{equation}
%   \#\ \text{s}\textbf{C} \Rightarrow \text{es}.\textbf{C}
% \end{equation}

% \paragraph{Syncope}

% \begin{equation}
%   \textbf{CV.C} \Rightarrow \textbf{CC}
% \end{equation}

% \paragraph{Betacism}

% \begin{align}
% \text{w} & \Rightarrow \text{\textipa{B}} \\
% \text{b} & \Rightarrow \text{\textipa{B}}
% \end{align}

% \paragraph{Diphthongs from Betacism}

% \begin{equation}
% \textbf{V}\text{\textipa{B}} \Rightarrow \textbf{V}\text{w}
% \end{equation}

% \paragraph{Deaspiration in Popular Latin}

% \begin{equation}
%   \text{h} \Rightarrow \varnothing
% \end{equation}

% \paragraph{Cluster Reduction in /ns/}

% \begin{equation}
%   \text{ns} \Rightarrow \text{s}
% \end{equation}

% \paragraph{Elision of intervocalic /g/}

% \begin{equation}
%   \textbf{V}.\text{g}\textbf{V} \Rightarrow \textbf{VV}
% \end{equation}

% \paragraph{Elision of coda /m/}

% \begin{equation}
%   \text{m}\ \# \Rightarrow \varnothing
% \end{equation}

% \paragraph{Degemination}

% \begin{equation}
%   \textbf{CC} \Rightarrow \textbf{C}
% \end{equation}

% \paragraph{Intervocalic Lenition}

% \begin{align}
%   \textbf{V.C}_{[-\text{voice}]}\textbf{V} & \Rightarrow \textbf{V.C}_{[+\text{voice}]}\textbf{V} \\
%   \textbf{V.C}\substack{[+\text{voice}] \\ [+\text{stop}]}\textbf{V} & \Rightarrow \textbf{V.C}\substack{[+\text{voice}] \\ [+\text{fricative}]}\textbf{V}
% \end{align}

% \paragraph{Debuccalization of onset /f/}

% \begin{equation}
%   \#\ \text{f} \Rightarrow \text{h}
% \end{equation}

% \paragraph{Fortition of onset /j/}

% \begin{equation}
%   \#\ \text{j} \Rightarrow \text{\textipa{\textdyoghlig}}
% \end{equation}

% \paragraph{Palatalization of Consonant Clusters}

% \begin{align}
%   \text{kt} & \Rightarrow \text{it} \\
%   \text{kl} & \Rightarrow \text{\textipa{L}} \\
%   \text{n:} & \Rightarrow \text{\textipa{\textltailn}} \\
%   \text{l:} & \Rightarrow \text{\textipa{L}} \\
%   \text{di} & \Rightarrow \text{\textipa{\textdyoghlig}} \\
%   \text{gi} & \Rightarrow \text{\textipa{\textdyoghlig}}
% \end{align}

% \paragraph{Palatalization of Velars}

% \begin{equation}
%   \textbf{C}_{[+\text{velar}]}\textbf{V}_{[+\text{front}]} \Rightarrow \textbf{C}_{[+\text{palatal}]}\textbf{V}_{[+\text{front}]}
% \end{equation}


% \paragraph{Palatalization of Dentals}

% \begin{equation}
%   \textbf{C}\substack{[+\text{dental}] \\ [+\text{stop}]}\text{i} \Rightarrow \textbf{C}\substack{[+\text{dental}] \\ [+\text{affricate}]}
% \end{equation}

% \paragraph{Palatalization of /n/ and /j/}

% \begin{align}
%   \text{nj} & \Rightarrow \text{\textipa{\textltailn}} \\
%   \text{lj} & \Rightarrow \text{\textipa{L}}
% \end{align}

% \paragraph{Palatalization of /Cl/ Clusters}

% \begin{align}
%   \#\ \textbf{C}\text{l} & \Rightarrow \text{\textipa{L}} \\
%   \text{-}\textbf{C}\text{l-} & \Rightarrow \begin{cases}
%                                               \text{\textipa{\textteshlig}} \\
%                                               \text{\textipa{\textdyoghlig}} \\
%                                               \text{\textipa{\textltailn}} \\
%                                               \text{\textipa{L}}
%                                             \end{cases}
% \end{align}

% \paragraph{Evolution of /skl/, /sl/, and /tl/}

% \begin{align}
%   \text{skl} & \Rightarrow \begin{cases}
%                              \text{\textipa{\texttslig}kl} \\
%                              \text{\textipa{\textteshlig}}
%                            \end{cases}\\
%   \text{sl} & \Rightarrow \text{skl} \\
%   \text{tl} & \Rightarrow \text{kl}
% \end{align}

% \chapter{Anthology of Morphophonological Changes}

% \section{Nominal}

% \subsection{Dissolution of Declension IV \& V}

% \subsection{Reduction of the Case System}

% \section{Verbal}

% \chapter{Glossarium Latinocastellanum}

% \section{Nominal}

% \subsection{Declension I \& V}

% \begin{tabular}{|c|c|}
%   \hline
%   \textsc{latina} & Español \\
%   \hline
%   \textsc{capra} & cabra \\
%   \hline
% \end{tabular}

% \begin{tabular}{c c}
%   \begin{tabular}{|c|c|c|}
%     \hline
%     & \textsc{sg} & \textsc{pl} \\
%     \hline
%     \textsc{nom} & \textsc{capra }& \textsc{caprae} \\
%     \hline
%     \textsc{gen} & \textsc{caprae} & \textsc{capr\={a}rum} \\
%     \hline
%     \textsc{dat} & \textsc{caprae} & \textsc{capr\={i}s} \\
%     \hline
%     \textsc{acc} & \cellcolor{gray} \textsc{capram} & \cellcolor{magenta} \textsc{capr\={a}s} \\
%     \hline
%     \textsc{abl} & \textsc{capr\={a}} & \textsc{capr\={i}s} \\
%     \hline
%     \textsc{voc} & \textsc{capra} & \textsc{caprae} \\
%     \hline
%   \end{tabular} \quad $\Rightarrow$ &

%   \begin{tabular}{|c|c|c|}
%     \hline
%     & \textsc{sg} & \textsc{pl} \\
%     \hline
%     & \cellcolor{gray} cabra & \cellcolor{magenta} cabras \\
%     \hline
%   \end{tabular} \\
% \end{tabular}

% \begin{tabular}{|c|c|}
%   \hline
%   \textsc{latina} & Español \\
%   \hline
%   \textsc{dies} & día \\
%   \hline
% \end{tabular}

% \subsection{Declension II \& IV}

% \begin{tabular}{|c|c|}
%   \hline
%   \textsc{latina} & Español \\
%   \hline
%   \textsc{murus} & muro \\
%   \hline
% \end{tabular}

% \begin{tabular}{|c|c|}
%   \hline
%   \textsc{latina} & Español \\
%   \hline
%   \textsc{fructus} & fruto\footnote{Old Spanish \emph{frucho}} \\
%   \hline
% \end{tabular}

% \subsection{Declension III}

% \begin{tabular}{|c|c|}
%   \hline
%   \textsc{latina} & Español \\
%   \hline
%   \textsc{panis} & pan \\
%   \hline
% \end{tabular}

% \section{Pronoun}

% \begin{tabular}{|c|c|}
%   \hline
%   \textsc{latina} & Español \\
%   \hline
%   \textsc{ego} & yo \\
%   \hline
%   \textsc{tv} & tú \\
%   \hline
%   \textsc{nos} & nosotros\footnote{Old Spanish \emph{nos}} \\
%   \hline
%   \textsc{vos} & vosotros\footnote{Old Spanish \emph{vos}} \\
%   \hline
%   \textsc{ille} & él \\
%   \hline
%   \textsc{illi} & ellos \\
%   \hline
% \end{tabular}

% \section{Numeral}

% \begin{tabular}{|c|c|}
%   \hline
%   \textsc{latina} & Español \\
%   \hline
%   \textsc{vnvs} & uno \\
%   \hline
%   \textsc{dvo} & dos \\
%   \hline
%   \textsc{tres} & tres \\
%   \hline
%   \textsc{qvattvor} & cuatro \\
%   \hline
%   \textsc{qvinqve} & cinco \\
%   \hline
%   \textsc{sex} & seis \\
%   \hline
%   \textsc{septem} & siete \\
%   \hline
%   \textsc{octo} & ocho \\
%   \hline
%   \textsc{novem} & nueve \\
%   \hline
%   \textsc{decem} & diez \\
%   \hline
% \end{tabular}

% \section{Verbal}

% \subsection{Conjugation I}

% \begin{tabular}{|c|c|}
%   \hline
%   \textsc{latina} & Español \\
%   \hline
%   \textsc{pensare} & pensar \\
%   \hline
% \end{tabular}

% \subsection{Conjugation II}

% \begin{tabular}{|c|c|}
%   \hline
%   \textsc{latina} & Español \\
%   \hline
%   \textsc{habere} & haber \\
%   \hline
% \end{tabular}

% \subsection{Conjugation III}

% \begin{tabular}{|c|c|}
%   \hline
%   \textsc{latina} & Español \\
%   \hline
%   \textsc{perdere} & perder \\
%   \hline
% \end{tabular}

% \subsection{Conjugation IV}

% \begin{tabular}{|c|c|}
%   \hline
%   \textsc{latina} & Español \\
%   \hline
%   \textsc{partire} & partir \\
%   \hline
% \end{tabular}

% \chapter*{Epilogue: \emph{Mein liebster Jugendtraum}}
% \addcontentsline{toc}{chapter}{Epilogue: \emph{Mein liebester Jugendtraum}}

% \epigraph{En adelanto van estos lugares: \\ ya tienen su diosa coronada.}{Leandro Díaz}

\nocite{*}

\chapter*{Bibliography}
\addcontentsline{toc}{chapter}{Biblipgraphy}

\printbibliography[heading=subbibintoc, keyword=phone, title={Phonology}]

\printbibliography[heading=subbibintoc, keyword=romance, title={Romance Linguistics}]

\printbibliography[heading=subbibintoc, keyword=tcs, title={Computer Science}]

\end{document}
