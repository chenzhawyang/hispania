\documentclass{report}

\usepackage{geometry}
\usepackage{appendix}[toc]
\usepackage{xeCJK}
\setCJKmainfont{Source Han Serif}
\usepackage{inputenc}[utf-8]
\usepackage{fontspec}
\setmainfont{Noto Serif}
\usepackage{babel}[USenglish]
\usepackage{minted}
\usepackage{bussproofs}
\usepackage{qtree}
\usepackage{amsmath}
\usepackage{amssymb}
\usepackage{amsthm}
\usepackage{tikz-cd}
\usepackage{tipa}
\usepackage{listings}
\usepackage[backend=biber, sorting=none, style=alphabetic]{biblatex}
\addbibresource{hispania.bib}
\usepackage{xcolor}
\usepackage{yfonts}
\usepackage{epigraph}
\usepackage{tcolorbox}
\usepackage{colortbl}
\usepackage{mathtools}
\usepackage{stmaryrd}
\usepackage{fourier}
\usepackage{footnote}
\makesavenoteenv{tabular}
\makesavenoteenv{table}
\usepackage{titlesec}
\renewcommand{\thechapter}{\Roman{chapter}}
\titleformat{\chapter}[display]{\huge\bf}{\textsc{Capitulum} \thechapter}{1em}{}
\usepackage{tocloft}
\setlength\cftchapnumwidth{3em}
\setlength\cftsecnumwidth{3em}
\setlength\cftsubsecnumwidth{4em}

\newtheorem{definition}{Definition}
\newtheorem{theorem}{Theorem}
\newtheorem{lemma}{Lemma}
\newtheorem{remark}{Remark}

\setlength\parindent{0pt}
\pagenumbering{Roman}

\usepackage{hyperref}
\hypersetup{
  colorlinks=true,
  linkcolor=blue,
  filecolor=magenta,
}

\title{Semi-Automated Language Changes from Latin to Spanish \\ or, \emph{El sueño de mi juventud: Fonología automatizada}}
\author{陳朝陽 \\ Chen Zhaoyang \\ \texttt{zc23@illinois.edu}}

\begin{document}

\maketitle

\pagebreak

\begin{abstract}
  
  This paper talks about an implementation of rule-based phonology in a high-level programming language. To illustrate the competence of the system, the system is used to implement major sound changes in the historical evolution of Latin to Modern Spanish. To a lesser extent, morphophonological changes from Latin to Spanish are also experimented. \\
  In the first couple chapters of this paper, we establish notations that represent the phonology and morphology of Latin and its Romance descendants following the tradition and convention of modern rule-based phonology that started to develop in the 1960s. After that we describe the major phonological and morphological changes the predecessors of Spanish had undergone, in the following historical periods: Proto-Romance, Western Romance, Central Iberian Romance, Old Spanish, Early Modern Spanish, and eventually we would be able to derive Modern Spanish. \\
  After settling the chronology and defining historical changes of Spanish language, in the last part of this project we explore the details of the implementation of a rewriting system that is able to substantiate and automate much of the aforementioned language changes.

\end{abstract}

\pagebreak

\tableofcontents

\pagebreak

\chapter*{Introduction}
\addcontentsline{toc}{chapter}{Introduction}

\epigraph{枯れた技術の水平思考\footnotemark}{横井軍平}
\footnotetext{\href{https://en.wikipedia.org/wiki/Gunpei_Yokoi\#Lateral_Thinking_with_Withered_Technology}{``Lateral Thinking of Withered Technology.''}}

\paragraph{Rule-based Phonology}

\paragraph{Romance Historical Linguistics}

\paragraph{Typed Programming Languages}

\chapter{Representation of Segments}

\section{Vocalic}

\begin{lstlisting}[basicstyle=\ttfamily, mathescape, escapeinside={(:}{:)}]
  vowel ::= height (:$\times$:) centrality (:$\times$:) duration

  height ::= low | low-mid | mid | high-mid | high

  centrality ::= front | central | back

  duration ::= short | long
\end{lstlisting}

\begin{minted}[autogobble]{hs}
  data Vowel = Vowel
  { height :: Height
    , centrality :: Centrality
    , dure :: Duration
  }

  data Height
  = Low
  | LoMid
  | Mid
  | HiMid
  | High

  data Centrality
  = Front
  | Cent
  | Back

  data Duration
  = Short
  | Long
\end{minted}

\section{Consonantal}

\begin{lstlisting}[basicstyle=\ttfamily, mathescape, escapeinside={(:}{:)}]
  consonant ::= manner (:$\times$:) place (:$\times$:) voice

  manner ::= nasal | stop | fricative | affricate
           | approximant | tap | trill | lateral

  place ::= bilabial
          | dental
          | palatal
          | palatoalvelar | velar | labialvelar
          | glottal

  place ::= voiced | voiceless
\end{lstlisting}

\chapter{Representation of Syllables}

\section{Onset}

\subsection{Onset Cluster}

\section{Nucleus}

\section{Coda}

\subsection{Coda Cluster}

\section{Syllable Weight and Stress}

\chapter{Representation of Morphophonology}

\section{Nominal}

\subsection{Declensions}

\subsection{Cases}

\section{Verbal}

\subsection{Persons}

\subsection{Tense}

\subsection{Aspect}

\subsection{Mood}

\subsection{Voice}

\chapter{Representation of Phonotactics and Sound Changes}

\epigraph{星翳燈幻露泡夢電雲\footnotemark}{金剛經 \S32}
\footnotetext{\href{https://www2.hf.uio.no/polyglotta/index.php?page=record\&vid=1176\&mid=1993243}{A shooting star, a clouding of the sight, a lamp, an illusion, a drop of dew, a bubble, a dream, a lightning’s flash, a thunder cloud.}}
\section{Sound Changes}

\section{Phonotactics}

\chapter{Proto-Romance}

\epigraph{\textsc{finis origine pendent}}{\textsc{marcvs manilivs}}

\section{Phonemic Inventory}

\subsection{Vocalic}

\begin{tcolorbox}[hbox, title=Proto-Romance Vowels after the Great Merger]
  \begin{tabular}{|c|c|c|c|}
    \hline
    & Front & Central & Back \\
    \hline
    High & i & & u \\
    \hline
    Mid & e & & o \\
    \hline
    Low-Mid & \textipa{E} & & \textipa{O} \\
    \hline
    Low & & a & \\
    \hline
  \end{tabular}
\end{tcolorbox}

\subsection{Consonantal}

\begin{tcolorbox}[title=Proto-Romance Consonants]
  
\end{tcolorbox}

\section{Sound Changes}

\subsection{Vocalism}

\begin{tabular}{c c}
  \begin{tabular}{|c|c|c|c|}
    \hline
    & Front & Central & Back \\
    \hline
    High & \textsc{\u{i} \={i}} & & \textsc{\u{u} \={u}} \\
    \hline
    Mid & \textsc{\u{e} \={e}} & & \textsc{\u{o} \={o}} \\
    \hline
    Low & & \textsc{\u{a} \={a}} & \\
    \hline
  \end{tabular}
  \quad $\Rightarrow$ & 
  \begin{tabular}{|c|c|c|c|}
    \hline
    & Front & Central & Back \\
    \hline
    High & i & & u \\
    \hline
    High-Mid & \cellcolor{gray} \textipa{I} & & \cellcolor{gray} \textipa{U} \\
    \hline
    Mid & e & & o \\
    \hline
    Low-Mid & \textipa{E} & & \textipa{O} \\
    \hline
    Low & & a & \\
    \hline
  \end{tabular}
  \vspace{0.3cm} \\
\end{tabular}

\subsection{Consonantism}

\begin{tcolorbox}[hbox, title=Latin Consonants]
  \begin{tabular}{|c|c|c|c|c|c|c|}
    \hline
    & Bilabial & Dental & Palatal & Velar & Labialvelar & Glottal \\
    \hline
    Nasal & m & n & & & & \\
    \hline
    Stop & p \quad b & t \quad d & & k \quad g & \textipa{k\super w} \quad \textipa{g\super w} & \\
    \hline
    Fricative & f & s \quad z & & & & \cellcolor{gray} h \\
    \hline
    Approximant & & & j & & w & \\
    \hline
    Trill & & r & & & & \\
    \hline
    Lateral & & l & & & & \\
    \hline
  \end{tabular}
\end{tcolorbox}

\section{Morphophonology}

\subsection{Nominal}

\subsection{Verbal}

\chapter{Western Romance}

\epigraph{\textsc{venite igitur descendamvs et confvndamvs ibi lingvam eorvm vt non avdiat vnvsqvisqve vocem proximi svi}}{\textsc{genesis} \textsc{xi}$\bullet$\textsc{vii}}

\section{Phonemic Inventory}

\subsection{Vocalic}

\subsection{Consonantal}

\section{Sound Changes}

\subsection{Vocalism}

\subsection{Consonantalism}

\section{Morphophonology}

\subsection{Nominal}

\subsection{Verbal}

\chapter{Central Iberian Romance}

\epigraph{Yo te aseguro que hoy estarás conmigo en el paraíso.}{Lucas 23:34}

\section{Phonemic Inventory}

\subsection{Vocalic}

\subsection{Consonantal}

\section{Sound Changes}

\subsection{Vocalism}

\subsection{Consonantalism}

\section{Morphophonology}

\subsection{Nominal}

\subsection{Verbal}

\chapter{Old Spanish}

\epigraph{Ya lo vedes, que partirnos emos en vida, yo iré e vós fincaredes remanida.}{Cantar de Mio Cid}

\section{Phonemic Inventory}

\subsection{Vocalic}

\subsection{Consonantal}

\section{Sound Changes}

\subsection{Vocalism}

\subsection{Consonantalism}

\section{Morphophonology}

\subsection{Nominal}

\subsection{Verbal}

\chapter{Early Modern Spanish}

\epigraph{El que sirve una revolución ara en el mar.}{Simón Bolívar}

\section{Phonemic Inventory}

\subsection{Vocalic}

\subsection{Consonantal}

\section{Sound Changes}

\subsection{Vocalism}

\subsection{Consonantalism}

\section{Morphophonology}

\subsection{Nominal}

\subsection{Verbal}

\chapter{Modern Spanish}

\epigraph{Todo está cumplido.}{Juan 19:30}

\section{Phonemic Inventory}

\subsection{Vocalic}

\subsection{Consonantal}

\section{Sound Changes}

\subsection{Vocalism}

\subsection{Consonantalism}

\section{Morphophonology}

\subsection{Nominal}

\subsection{Verbal}

\chapter{Anthology of Sound Changes}

\epigraph{How many years must a mountain exist \\ Before it is washed to the sea?}{Bob Dylan}

\section{Vocalism}

\begin{tcolorbox}[hbox, title=Latin]
  \begin{tabular}{|c|c|c|c|}
    \hline
    & Front & Central & Back \\
    \hline
    High & \textsc{\u{i} \={i}} & & \textsc{\u{u} \={u}} \\
    \hline
    Mid & \textsc{\u{e} \={e}} & & \textsc{\u{o} \={o}} \\
    \hline
    Low & & \textsc{\u{a} \={a}} & \\
    \hline
  \end{tabular}
\end{tcolorbox} 

\begin{tcolorbox}[hbox, title=Proto-Romance]
  \begin{tabular}{|c|c|c|c|}
    \hline
    & Front & Central & Back \\
    \hline
    High & i & & u \\
    \hline
    High-Mid & \cellcolor{gray} \textipa{I} & & \cellcolor{gray} \textipa{U} \\
    \hline
    Mid & e & & o \\
    \hline
    Low-Mid & \textipa{E} & & \textipa{O} \\
    \hline
    Low & & a & \\
    \hline
  \end{tabular}
\end{tcolorbox}

\begin{tcolorbox}[hbox, title=Proto-Romance after the Great Merger]
  \begin{tabular}{|c|c|c|c|}
    \hline
    & Front & Central & Back \\
    \hline
    High & i & & u \\
    \hline
    Mid & e & & o \\
    \hline
    Low-Mid & \cellcolor{magenta} \textipa{E} & & \cellcolor{magenta} \textipa{O} \\
    \hline
    Low & & a & \\
    \hline
  \end{tabular}
\end{tcolorbox}

\subsection{Monophthongs}

\begin{tcolorbox}[title=Proto-Romance Loss of Quantity]  
  \begin{equation}\tag{Loss of Quantity}\label{loss_of_quantity}
    \textbf{V}
    \begin{cases}
      \vdash \text{long} : \textsc{duration} & \Rightarrow [\text{long} \rightarrow \text{short}]\ \textbf{V} \\
      \vdash \text{short} : \textsc{duration} & \Rightarrow
                                                \begin{cases}
                                                  \textsc{a} & \Rightarrow \textbf{V} \\
                                                  \texttt{else} & \Rightarrow \begin{cases}
                                                                              \vdash \text{high} : \textsc{height} & \Rightarrow [\text{high} \rightarrow \text{high-mid}]\ \textbf{V} \\
                                                                              \vdash \text{mid} : \textsc{height} & \Rightarrow [\text{mid} \rightarrow \text{low-mid}]\ \textbf{V} \\
                                                                            \end{cases} \\
                                                \end{cases} \\
    \end{cases}
  \end{equation}  
\end{tcolorbox}

\begin{tcolorbox}[title=Pan-Romance Great Merger]
  \begin{equation}\tag{Great Merger}\label{eq:great_merger}
    \textbf{V}_{\vdash \text{high-mid}} \Rightarrow [\text{high-mid} \rightarrow \text{mid}]\ \textbf{V}
  \end{equation}
\end{tcolorbox}

\begin{tcolorbox}[title=Monophthongization]
  \begin{align*}\tag{Monophthongization}\label{eq:monophthongization}
    \textsc{oe} & \Rightarrow \text{e} \\
    \textsc{au} & \Rightarrow \text{o} \\
    \textsc{ae} & \Rightarrow \text{\textipa{E}} 
  \end{align*}
\end{tcolorbox}

\begin{tcolorbox}[title=Diphthongization]
  \begin{align*}\tag{Diphthongization}\label{eq:diphthongization}
      \text{\textipa{E}} \vdash \textsc{stressed} & \Rightarrow \text{ie} \\
      \text{\textipa{O}} \vdash \textsc{stressed} & \Rightarrow \text{we}
  \end{align*}
\end{tcolorbox}

\subsection{Diphthongs}

\begin{tcolorbox}[title=Loss of Hiatus]
  \begin{equation}\tag{Loss of Hiatus}\label{eq:loss_of_hiatus}
    \textbf{V.V} \Rightarrow \textbf{VV}
  \end{equation}
\end{tcolorbox}

\subsection{Metaphony}

\section{Consonantism}

\subsection{Monolithic Consonants}

\subsection{Consonant Clusters}

\paragraph{Prothesis in /sC/}

\begin{equation}
  \#\ \text{s}\textbf{C} \Rightarrow \text{es}.\textbf{C}
\end{equation}

\paragraph{Syncope}

\begin{equation}
  \textbf{CV.C} \Rightarrow \textbf{CC}
\end{equation}

\paragraph{Betacism}

\begin{align}
\text{w} & \Rightarrow \text{\textipa{B}} \\
\text{b} & \Rightarrow \text{\textipa{B}}
\end{align}

\paragraph{Diphthongs from Betacism}

\begin{equation}
\textbf{V}\text{\textipa{B}} \Rightarrow \textbf{V}\text{w}
\end{equation}

\paragraph{Deaspiration in Popular Latin}

\begin{equation}
  \text{h} \Rightarrow \varnothing
\end{equation}

\paragraph{Cluster Reduction in /ns/}

\begin{equation}
  \text{ns} \Rightarrow \text{s}
\end{equation}

\paragraph{Elision of intervocalic /g/}

\begin{equation}
  \textbf{V}.\text{g}\textbf{V} \Rightarrow \textbf{VV}
\end{equation}

\paragraph{Elision of coda /m/}

\begin{equation}
  \text{m}\ \# \Rightarrow \varnothing
\end{equation}

\paragraph{Degemination}

\begin{equation}
  \textbf{CC} \Rightarrow \textbf{C}
\end{equation}

\paragraph{Intervocalic Lenition}

\begin{align}
  \textbf{V.C}_{[-\text{voice}]}\textbf{V} & \Rightarrow \textbf{V.C}_{[+\text{voice}]}\textbf{V} \\
  \textbf{V.C}\substack{[+\text{voice}] \\ [+\text{stop}]}\textbf{V} & \Rightarrow \textbf{V.C}\substack{[+\text{voice}] \\ [+\text{fricative}]}\textbf{V}
\end{align}

\paragraph{Debuccalization of onset /f/}

\begin{equation}
  \#\ \text{f} \Rightarrow \text{h}
\end{equation}

\paragraph{Fortition of onset /j/}

\begin{equation}
  \#\ \text{j} \Rightarrow \text{\textipa{\textdyoghlig}}
\end{equation}

\paragraph{Palatalization of Consonant Clusters}

\begin{align}
  \text{kt} & \Rightarrow \text{it} \\
  \text{kl} & \Rightarrow \text{\textipa{L}} \\
  \text{n:} & \Rightarrow \text{\textipa{\textltailn}} \\
  \text{l:} & \Rightarrow \text{\textipa{L}} \\
  \text{di} & \Rightarrow \text{\textipa{\textdyoghlig}} \\
  \text{gi} & \Rightarrow \text{\textipa{\textdyoghlig}}
\end{align}

\paragraph{Palatalization of Velars}

\begin{equation}
  \textbf{C}_{[+\text{velar}]}\textbf{V}_{[+\text{front}]} \Rightarrow \textbf{C}_{[+\text{palatal}]}\textbf{V}_{[+\text{front}]}
\end{equation}


\paragraph{Palatalization of Dentals}

\begin{equation}
  \textbf{C}\substack{[+\text{dental}] \\ [+\text{stop}]}\text{i} \Rightarrow \textbf{C}\substack{[+\text{dental}] \\ [+\text{affricate}]}
\end{equation}

\paragraph{Palatalization of /n/ and /j/}

\begin{align}
  \text{nj} & \Rightarrow \text{\textipa{\textltailn}} \\
  \text{lj} & \Rightarrow \text{\textipa{L}}
\end{align}

\paragraph{Palatalization of /Cl/ Clusters}

\begin{align}
  \#\ \textbf{C}\text{l} & \Rightarrow \text{\textipa{L}} \\
  \text{-}\textbf{C}\text{l-} & \Rightarrow \begin{cases}
                                              \text{\textipa{\textteshlig}} \\
                                              \text{\textipa{\textdyoghlig}} \\
                                              \text{\textipa{\textltailn}} \\
                                              \text{\textipa{L}}
                                            \end{cases}
\end{align}

\paragraph{Evolution of /skl/, /sl/, and /tl/}

\begin{align}
  \text{skl} & \Rightarrow \begin{cases}
                             \text{\textipa{\texttslig}kl} \\
                             \text{\textipa{\textteshlig}}
                           \end{cases}\\
  \text{sl} & \Rightarrow \text{skl} \\
  \text{tl} & \Rightarrow \text{kl}
\end{align}

\chapter{Anthology of Morphophonological Changes}

\section{Nominal}

\subsection{Dissolution of Declension IV \& V}

\subsection{Reduction of the Case System}

\section{Verbal}

\chapter{Glossarium}

\section{Nominal}

\subsection{Declension I \& V}

\begin{tabular}{|c|c|}
  \hline
  \textsc{latina} & Español \\
  \hline
  \textsc{capra} & cabra \\
  \hline
\end{tabular}

\begin{tabular}{c c}
  \begin{tabular}{|c|c|c|}
    \hline
    & \textsc{sg} & \textsc{pl} \\
    \hline
    \textsc{nom} & \textsc{capra }& \textsc{caprae} \\
    \hline
    \textsc{gen} & \textsc{caprae} & \textsc{capr\={a}rum} \\
    \hline
    \textsc{dat} & \textsc{caprae} & \textsc{capr\={i}s} \\
    \hline
    \textsc{acc} & \cellcolor{gray} \textsc{capram} & \cellcolor{magenta} \textsc{capr\={a}s} \\
    \hline
    \textsc{abl} & \textsc{capr\={a}} & \textsc{capr\={i}s} \\
    \hline
    \textsc{voc} & \textsc{capra} & \textsc{caprae} \\
    \hline
  \end{tabular} \quad $\Rightarrow$ &

  \begin{tabular}{|c|c|c|}
    \hline
    & \textsc{sg} & \textsc{pl} \\
    \hline
    \textsc{nom} & \cellcolor{gray} cabra & \cellcolor{magenta} cabras \\
    \hline
  \end{tabular} \\
\end{tabular}

\begin{tabular}{|c|c|}
  \hline
  \textsc{latina} & Español \\
  \hline
  \textsc{dies} & día \\
  \hline
\end{tabular}

\subsection{Declension II \& IV}

\begin{tabular}{|c|c|}
  \hline
  \textsc{latina} & Español \\
  \hline
  \textsc{murus} & muro \\
  \hline
\end{tabular}

\begin{tabular}{|c|c|}
  \hline
  \textsc{latina} & Español \\
  \hline
  \textsc{fructus} & fruto\footnote{Old Spanish \emph{frucho}} \\
  \hline
\end{tabular}

\subsection{Declension III}

\begin{tabular}{|c|c|}
  \hline
  \textsc{latina} & Español \\
  \hline
  \textsc{panis} & pan \\
  \hline
\end{tabular}

\section{Pronoun}

\begin{tabular}{|c|c|}
  \hline
  \textsc{latina} & Español \\
  \hline
  \textsc{ego} & yo \\
  \hline
  \textsc{tv} & tú \\
  \hline
  \textsc{nos} & nosotros\footnote{Old Spanish \emph{nos}} \\
  \hline
  \textsc{vos} & vosotros\footnote{Old Spanish \emph{vos}} \\
  \hline
  \textsc{ille} & él \\
  \hline
  \textsc{illi} & ellos \\
  \hline
\end{tabular}

\section{Numeral}

\begin{tabular}{|c|c|}
  \hline
  \textsc{latina} & Español \\
  \hline
  \textsc{vnvs} & uno \\
  \hline
  \textsc{dvo} & dos \\
  \hline
  \textsc{tres} & tres \\
  \hline
  \textsc{qvattvor} & cuatro \\
  \hline
  \textsc{qvinqve} & cinco \\
  \hline
  \textsc{sex} & seis \\
  \hline
  \textsc{septem} & siete \\
  \hline
  \textsc{octo} & ocho \\
  \hline
  \textsc{novem} & nueve \\
  \hline
  \textsc{decem} & diez \\
  \hline
\end{tabular}

\section{Verbal}

\subsection{Conjugation I}

\begin{tabular}{|c|c|}
  \hline
  \textsc{latina} & Español \\
  \hline
  \textsc{pensare} & pensar \\
  \hline
\end{tabular}

\subsection{Conjugation II}

\begin{tabular}{|c|c|}
  \hline
  \textsc{latina} & Español \\
  \hline
  \textsc{habere} & haber \\
  \hline
\end{tabular}

\subsection{Conjugation III}

\begin{tabular}{|c|c|}
  \hline
  \textsc{latina} & Español \\
  \hline
  \textsc{perdere} & perder \\
  \hline
\end{tabular}

\subsection{Conjugation IV}

\begin{tabular}{|c|c|}
  \hline
  \textsc{latina} & Español \\
  \hline
  \textsc{partire} & partir \\
  \hline
\end{tabular}

\chapter*{Epilogue: \emph{Mein liebster Jugendtraum}}
\addcontentsline{toc}{chapter}{Epilogue: \emph{Mein Jugendtraum}}

\nocite{*}

\chapter*{Bibliography}
\addcontentsline{toc}{chapter}{Biblipgraphy}

\printbibliography[heading=subbibintoc, keyword=romance, title={Romance Linguistics}]

\end{document}
